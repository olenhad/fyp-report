% Chapter 3

\chapter{Choosing the stack} % Chapter title

\label{ch:stack} % For referencing the chapter elsewhere, use \autoref{ch:mathtest}

This decision took a while. And rightfully so, as it would affect all decisions from that point. Choosing the right language and framework could have more than substantial effects on the system's architecture.\\

After quite a bit of research and debate, I decided to implement the system using \spacedlowsmallcaps{clojure}, a relatively new lisp class language than runs on the JVM.

\section{Clojure?}

The choice of implementing this project in clojure may seem peculiar. However there are a number of features that clojure and it's ecosystem that fit very well with this particular project. Before jumping to those features, a quick overview of clojure is in order.\\

Clojure is a lisp dialect, created by Rich Hickey. It is a general-purpose, functional programming language that runs on the Java Virtual Machine (JVM) \footnote{Clojure runs in other environments like Microsoft's Common Language Runtime (CLR) and Javascript engines as well}. Clojure focuses on programming with immutable data structures and provides a number of persistent hash-trie based data structures that allow efficient functional programming.\\

Clojure was designed to be a real world language, as opposed to a purely academic one.
