% Chapter 4

\chapter{Modelling The Problem} % Chapter title

\label{ch:model} % For referencing the chapter elsewhere, use \autoref{ch:name}

As mentioned before, the core function of the application was to convert periodic flight data into a
renderable 3d model. This section will detail the approach taken to model the various aspects of the flight data.\\


\section{Inputs}

The dataset AAIB provided, contained the following measurements, recorded over time t.
\begin{enumerate}
  \item Magnetic Heading $\theta_{m}(t)$. Measures the direction of the aircraft with respect to the north pole. Measured in degrees
  \item Pressure Altitude $a(t)$. Indicated altitude, measured in feet
  \item GPS Latitude $\phi(t)$. Measured in degrees
  \item GPS Longtitude $\lambda(t)$. Also measured in degrees
  \item Roll Angle $\theta_{r}(t)$. Also measured in degrees
  \item Engine Thrust $e_{i}(t), 0 < i \leq 4$. Measured as a percentage
\end{enumerate}

The dataset had other fields as well, but they were not used for the model implemented

\section{Constraints}

A constraint that choosing KML placed on the model was that any coordinate could only be represented withe following dimensions:
\begin{enumerate}
\item Latitude ($\phi$)
\item Longtitude ($\lambda$)
\item Altitude (a)
\end{enumerate}

This implied that plotting polygons with KML would require converting to and fro between cartesian and spherical coordinate systems.
%----------------------------------------------------------------------------------------

\section{Flight Path}

Plotting the flight path was the most straightforward operation.

%------------------------------------------------

\subsection{Subsection Title}

Content

%------------------------------------------------

\subsection{Subsection Title}

Content

%----------------------------------------------------------------------------------------

\section{Section Title}

Content
