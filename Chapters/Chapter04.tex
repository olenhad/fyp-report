% Chapter 4

\chapter{Modelling The Problem} % Chapter title

\label{ch:model} % For referencing the chapter elsewhere, use \autoref{ch:name}

As mentioned before, the core function of the application was to convert periodic flight data into a
renderable 3d model. This section will detail the approach taken to model the various aspects of the flight data.\\


\section{Inputs}

As mentioned before

\section{Constraints}

A constraint that choosing KML placed on the model was that any coordinate could only be represented withe following dimensions:
\begin{enumerate}
\item Latitude ($\phi$)
\item Longtitude ($\lambda$)
\item Altitude (a)
\end{enumerate}

This implied that plotting polygons with KML would require converting to and fro between cartesian and spherical coordinate systems.
%----------------------------------------------------------------------------------------

\section{Flight Path}

Plotting the flight path is the most straightforward

%------------------------------------------------

\subsection{Subsection Title}

Content

%------------------------------------------------

\subsection{Subsection Title}

Content

%----------------------------------------------------------------------------------------

\section{Section Title}

Content
