% Chapter 1

\chapter{Introduction} % Chapter title

\label{ch:introduction} % For referencing the chapter elsewhere, use \autoref{ch:introduction}

%----------------------------------------------------------------------------------------
\section{Background}

Aircraft systems record large quantities of data during flight. A lot of this data is necessary in the system 's proper functioning. It takes an even more significant role when investigating air accidents.
As this data establishes the causes behind those accidents, it aids measures to prevent such accidents from occurring. That is the primary reason for why recovery of Flight Data Recorders \footnote{Popularly referred to as `Blackboxes`} is usually a very high priority for aircraft investigations.\\

The data itself features many components tracked over time, including but not limited to the following\footnote{The following components were part of the sample dataset I was given for testing.}:
\begin{itemize}
\item Latitude
\item Longtitude
\item Altitude
\item Heading
\item Pitch
\item Roll
\item Engine Thrust
\item Throttle Lever Position
\end{itemize}

As is probably evident, analyzing this largely numeric data by itself would be a painstaking task. Primarily because finding patterns in numbers through observation is inherently difficult. Also because the data set is of a substantially large size \footnote{I will be cautious before calling it ``Big Data''. }. \\

Therefore the need for a tool to visualise this data becomes immediately obvious. \\

The Air Accident Investigation Board (AAIB), a branch of Singapore's Ministry of Transport performs analysis on such flight data, and requested an application that would render relevant flight data quickly and in a portable manner. \\

\section{System Requirements}

AAIB needed the application to meet certain criterias and contain certain core features. These included:

\begin{enumerate}
  \item \spacedlowsmallcaps{Google Earth} Compatibility. Google Earth is a popular ``virtual globe, map and geographic information program''. It provides detailed 3d projections of Google's satellite images on a spherical globe, overlaid with huge quantities of textual and pictorial data varying from international boundaries to streets. There are many good reasons for this requirement including:
    \begin{itemize}
      \item Cross Platform. It works and is supported on Windows, OSX and Linux \footnote{Atleast the desktop version is supported on all three major Operating Systems.}
      \item Extensible through the Keyhole Markup Language (KML) \footnote{A lot more on this later}
      \item Free\footnote{As in beer, not speech}
      \item AAIB's familiarity with it
      \item Offline use. This is quite important as internet access is not guranteed on accident sites
    \end{itemize}
  \item \spacedlowsmallcaps{CSV}\footnote{Comma Separated Values} based input. The system should be able to use large datasets encoded in CSV.
  \item  The system should display the data in its geographical location as vectors. The visualization should be able to reflect the actual data in an intuitive way. For instance, vectors at each data point should represent roll, heading, engine throttle
  \item Users should be able to select which vectors they want to view.
  \item The data points should be manually adjustable from within google earth. This is particularly relevant for data points near landing strips, where measurement errors may cause an offset in the rendered visualization

\end{enumerate}
